\documentclass[a4paper]{article}
\usepackage[utf8]{inputenc}
\usepackage[slovene]{babel}
\usepackage{graphicx}
\usepackage{hyperref}
\usepackage[nottoc]{tocbibind}
\usepackage{caption}
\usepackage{subcaption}
\usepackage{amsmath}
\usepackage{ dsfont }
\usepackage{siunitx}
\usepackage{multimedia}
\usepackage[table,xcdraw]{xcolor}
\setlength\parindent{0pt}

\usepackage{minted}
\usepackage{listings}
\usepackage{tcolorbox}
\tcbuselibrary{listings, minted, skins}

\newcommand{\ddd}{\mathrm{d}}
\newcommand\myworries[1]{\textcolor{red}{#1}}
\newcommand{\Dd}[3][{}]{\frac{\ddd^{#1} #2}{\ddd #3^{#1}}}

\newtcblisting{pythonlst}{listing only, minted language=python3, minted style=paraiso-dark,
    colback=bg, enhanced, frame hidden, minted options={, 
    fontsize=\footnotesize, tabsize=2, breaklines, autogobble}}

\definecolor{inline}{RGB}{187,57,82}
\definecolor{bg}{RGB}{22,43,58}
\setminted[python3]{bgcolor=bg, fontsize=\footnotesize}

\begin{document}
\begin{titlepage}
    \begin{center}
        \includegraphics[]{logo.png}
        \vspace*{3cm}
        
        \Huge
        \textbf{Linearno programiranje}
        
        \vspace{0.5cm}
        \large
        2. naloga pri Modelski analizi 1

        \vspace{4.5cm}
        
        \textbf{Avtor:} Marko Urbanč (28232019)\ \\
        \textbf{Predavatelj:} prof. dr. Simon Širca\ \\
        \textbf{Asistent:} doc. dr. Miha Mihovilovič\ \\
        
        \vspace{2.8cm}
        
        \large
        17.10.2023
    \end{center}
\end{titlepage}
\tableofcontents
\newpage
\section{Uvod}
Linearna optimizacija je zadnje čase zelo vroča tema, tako da me veseli, da sem se je lahko lotil tudi sam. 
Linearno programiranje oz. linearna optimizacija je metoda za iskanje maksimuma ali minimuma linearnega izraza,
ki je podvržen linearnim omejitvam. Omenjen linearni izraz je funkcija več spremenljivk, ki najpogosteje ovrednoti
oz. oceni neko količino, ki jo želimo optimizirati. To je lahko npr. dobiček, strošek, količina proizvodnje, itd.
zato to funkcijo imenujemo \texttt{cost function} oz. \texttt{objective function}. Linearno programiranje je torej
metoda, ki nam omogoča, da z linearno funkcijo in linearnimi omejitvami poiščemo optimalno rešitev. Uporablja v 
različnih panogah, kot so ekonomija, logistika, telekomunikacije, transport, kot pa tudi v znanosti, kot je npr. 
fizika. \\

Torej če si to pogledamo v matematični notaciji imamo našo cost funkcijo $f(x_1, x_2, \dots, x_n)$, 
definirano kot neko linearno kombinacijo spremenljivk:

\begin{equation}
    f(x_1, x_2, \dots, x_n) = c_1x_1 + c_2x_2 + \dots + c_nx_n\>.
\end{equation}

in set vezi, ki so pogosto izražene kot neenačbe:

\begin{equation}
    \begin{split}
        a_{11}x_1 + a_{12}x_2 + \dots + a_{1n}x_n &\leq b_1\>, \\
        a_{21}x_1 + a_{22}x_2 + \dots + a_{2n}x_n &\leq b_2\>, \\
        \vdots \qquad \qquad \quad\\
        a_{m1}x_1 + a_{m2}x_2 + \dots + a_{mn}x_n &\leq b_m\>. \\
    \end{split}
\end{equation}

To je pravzaprav to kar se tiče matematičnega opisa osnovnega problema. Seveda so potem izvedenke postopkov
oz. algoritmov malo bolj zapletene, ampak to žal ni namen te naloge. \\

Dotično pri tej nalogi si bomo pogledali problem optimizacije diete. Posebna omemba ni potrebna, kako 
pomembno je to, da se prehranjujemo zdravo in uravnoteženo. Vendar pa je to v današnjem času vse težje,
saj je na voljo ogromno različnih živil, ki so vse prej kot zdrava. Zato je toliko bolj pomembno, da
se zavedamo, kaj jemo in da se prehranjujemo zdravo. Vendar pa je to včasih težko, saj je veliko ljudi
prezaposlenih in nimajo časa, da bi se ukvarjali s tem. Skratka ideja je taka, da bomo za dane želje 
kar se tiče hranilnih snovi, poskusili najti optimalno dieto. Optimalno pa je tu mišljeno v tem smislu,
kot ga uporabnik določi. Zagotovo bo razlika med tem, ali se minimizira količina hrane, ki jo je potrebno
pojesti, cena hrane, količina določene hranilne snovi ali pa kalorična vrednost. 

\section{Naloga}
Naloga torej zahteva, da za dano tabelo živil in njihovih hranilnih vrednosti, najdemo optimalno dieto, kjer minimiziramo
kalorije, če zahtevamo sledeče:

\begin{enumerate}
    \item \begin{itemize}
        \item Vsaj 70 g maščob
        \item Vsaj 310 g ogljikovih hidratov
        \item Vsaj 50 g proteinov
        \item Vsaj 1000 mg kalcija
        \item Vsaj 18 mg železa
        \item Dnevni obroki naj ne presežejo 2000 g
    \end{itemize}

    Ta osnovni model lahko potem nadgradimo s tem da dodamo še:
    \begin{itemize}
        \item Vsaj 60 mg Vitamina C
        \item Vsaj 3500 mg Kalija
        \item Med 500 mg in 2400 mg Natrija
    \end{itemize}

    \item Potem pa poglejmo kako se rezultat razlikuje še zahtevamo vsaj $2000$ kcal in minimiziramo
    vnos maščob.

    \item Minimizirajmo zdaj ceno diete
    \item Dodatne izvedbe omejitev za izboljšave uravnoteženosti diete

\end{enumerate}

\section{Opis reševanja}
Reševanja sem se lotil v klasično v Pythonu. Pri tej nalogi je močno prišel v upoštev 
pythonian princip, da za vse že obstaja neka knjižnica. In res je tako, za reševanje linearne 
optimizacije obstaja knjižnica \texttt{PuLP}, ki je zelo enostavna za uporabo in ima še kar solidno 
dokumentacijo. Pravzaprav je njihov prvi demonstracijski zgled zelo podoben našemu, z razliko da oni
mešajo meso v mačji hrani. Poleg tega pa seveda pridejo zraven še standardni \texttt{numpy}, \texttt{pandas}
in tokrat zaradi malo alterative izbire vizualizacije še \texttt{holoviews}. \\


\subsection{\texttt{PuLP}: basic cookbook}
Kot sem že omenil, je knjižnica \texttt{PuLP} zelo enostavna za uporabo. Lahko na hitro povzamem postopek
ker tako ali tako trenutno čakam, da se mi predprocesirajo podatki za (hopefully uspe) zadnji del naloge.
Torej začnemo s tem da uvozimo knjižnico in definiramo problem. 

\begin{pythonlst}
    from pulp import *
    import pandas as pd

    # Create the "prob"  variable to contain the problem data
    prob = LpProblem("Diet Problem", LpMinimize)
\end{pythonlst}

Sledi nalaganje podatkov iz tabele živil in ustvarjanje spremenljivk. Spremenljivke so v tem primeru
količina živila, ki ga bomo pojedli. 

\begin{pythonlst}
    df = pd.read_table('Data/table.dat', sep=',', skiprows=2, index_col=0)

    # Create a list of the food items
    food_items = list(df.index)

    # Create variables. These variables are the amounts of each food item to buy
    food_vars = LpVariable.dicts("Food", food_items, lowBound=0, cat='Continuous')
\end{pythonlst}

Okay sedaj pa definiramo še našo cost funkcijo. V tem osnovnem primeru bo to količina kalorij, ki jih
bomo zaužili.

\begin{pythonlst}
    # Define objective function and add it to the problem
    prob += lpSum([df.loc[i, 'Energija[kcal]'] * food_vars[i] for i in food_items]), "Total energy intake per person"
\end{pythonlst}

Zelo praktično je, da v bistvu kar prištevamo izraze k našemu problemu. Tako lahko zelo enostavno dodajamo
vezi. Storimo to.

\begin{pythonlst}
    # And now we can add the constraints
    prob += lpSum([df.loc[i, 'Mascobe[g]'] * food_vars[i] for i in food_items]) >= 70, "FatRequirement"
    prob += lpSum([df.loc[i, 'Ogljikovi_Hidrati[g]'] * food_vars[i] for i in food_items]) >= 310, "CarbohydrateRequirement"
    prob += lpSum([df.loc[i, 'Proteini[g]'] * food_vars[i] for i in food_items]) >= 50, "ProteinRequirement"
    prob += lpSum([df.loc[i, 'Ca[mg]'] * food_vars[i] for i in food_items]) >= 1000, "CalciumRequirement"
    prob += lpSum([df.loc[i, 'Fe[mg]'] * food_vars[i] for i in food_items]) >= 18, "IronRequirement"
    prob += lpSum([100 * food_vars[i] for i in food_items]) <= 2000, "MassLimit"
\end{pythonlst}

Zdaj ko smo model tako lepo definirali je smiselno, da ga tudi spravimo v datoteko. Nato pa kličemo 
reševanje. \texttt{PuLP} bo sam poskrbel za izbiro algoritma, ki bo najbolj primeren za naš problem.

\begin{pythonlst}
    model_name = "diet-model_no-weight-con.lp"
    prob.writeLP(f"Models/{model_name}")

    # Slove the problem
    prob.solve()
\end{pythonlst}

In na koncu še spravimo rešitev v neko obliko, ki jo lahko potem uporabimo za vizualizacijo.

\begin{pythonlst}
    # Save the solution to a file with numpy
    var_names = np.array([v.name for v in prob.variables()])
    var_values = np.array([v.varValue for v in prob.variables()])
    solution = np.column_stack((var_names, var_values))

    np.savetxt(f"Solutions/{model_name}-sol.dat", solution, delimiter=",", fmt="\%s", header="Item,Value")
\end{pythonlst}

Če bi nas zanimalo še takojšnje ovrednotenje lahko z spodnjim blokom kode izpišemo nekaj uporabnih
informacij.

\begin{pythonlst}
    # Print the status of the solution
    print("Status:", LpStatus[prob.status])

    # Print the optimal solution
    for v in prob.variables():
        print(v.name, "=", v.varValue)
    print("Total energy intake per person = ", value(prob.objective))
\end{pythonlst}

Od tod naprej pa lahko poljubno zapletemo vezi (dokler ostanejo linearne seveda) ali pa spremenimo
vhodne podatke. \\

\section{Rezultati}
Tole poročilo bo res med krajšimi, kar sem jih napisal, kar se tiče surove vsebine, ampak 
dejstvo je, da je narava naloge taka, da gremo lahko takoj na zabavne rezultate. \\

\subsection{Osnovni model}
\subsection{Osnovni model brez omejitve mase}
Najprej si poglejmo rezultate osnovnega modela, kjer ne omejujemo mase. Torej lahko pojemo kolikor
želimo. Rezultati so prikazani na sliki (\ref{fig:basic-no-mass}). \\
\begin{figure}[H]
    \centering
    \makebox[\textwidth][c]{%
    \includegraphics[width=1.2\textwidth]{../Images/diet-model_basic-no-weight-visual.png}
    }
    \caption{Osnovni model brez omejitve mase}
    \label{fig:basic-no-mass}
\end{figure}

Tu smo dobili znanstveni dokaz, da nam radenska res da tri srčke. Presenetljivo številke niso tako
nerealistično velike kot bi lahko bile in kot bomo videli, da bodo, tako da bi dieta, ki bi slonela
na radenski, pomfriu, medu in kakavu lahko bila še kar obetavna. Radenska je fenomenalna izbira, ker 
optimiziramo kalorije, ki jih zaužijemo. Voda ie. radenska pa ima zelo malo kalorij, tako da je to
zelo dobra izbira, hkrati pa ima neko količino mineralov, ki jih potrebujemo in tako prevlada v izboru.

\subsubsection{Osnovni model z omejitvijo mase}
Sedaj pa si poglejmo še rezultate, kjer omejujemo maso na $2000$ g. Rezultati so prikazani na 
sliki (\ref{fig:basic}). \\

\begin{figure}[H]
    \centering
    \makebox[\textwidth][c]{%
    \includegraphics[width=1.2\textwidth]{../Images/diet-model_basic-visual.png}
    }
    \caption{Osnovni model z omejitve mase}
    \label{fig:basic}
\end{figure}

Meni se zdi neznosno amusing rezultat, da je glede na naše preproste vezi, optimalna dieta takšna, da se
pravzaprav ubiješ s soljo. Dodatno pa ješ kakav za mineralni vnos. To bi lahko bil jedilnik astronavtov v 
distopični prihodnosti, kjer smo nekako ugotovili, kako varno lahko jemo ogromne količine soli. Simpatično je,
da imamo tudi za okras in okus malo pomfrija. Mogoče so pa to your average \textit{Mc D's} pomfriji, kjer je
verjetno brez šale na $17$ g pomfrija $1.4$ kg soli. Sicer pa verjetno je tu argument podoben kot pri radenski.
Kar naenkrat rabimo nekaj kar je mineralno bolj gosto in kaj je boljše kot mineralna voda? Mineral sam. Tudi sol 
ima precej nizko kalorično vrednost, tako da je to zelo dobra izbira. O smrtnih posledicah hipernatriemije pa
ne bi.

\subsubsection{Osnovni model z omejitvijo mase in dodatnimi omejitvami}
Zdaj pa izključimo to zoprno sol in dodajmo še tiste dodatne omejitve, kjer si postavimo nek smiselen interval
za količino natrija v naši hrani. Ob tem pa lahko še dodamo par vezi, ker lahko. Rezultati so prikazani na
sliki (\ref{fig:basic-add}). \\

\begin{figure}[H]
    \centering
    \makebox[\textwidth][c]{%
    \includegraphics[width=1.2\textwidth]{../Images/diet-model_basic-add-visual.png}
    }
    \caption{Osnovni model z omejitve mase in dodatnimi omejitvami}
    \label{fig:basic-add}
\end{figure}

Opa, to pa je že nekaj kar bi lahko človek rekel, da je popolnoma passable. Dober liter radenske se ne zdi
pretiravanje, če te mehurčki ne motijo. Potem pa malo solate, malo sadja in seveda (tukaj odpove heh..) pol kile 
kakava. Očitno sta kakav in radenska še vedno zelo kalorično učinkovita.

\newpage
\subsection{Dieta z manj maščobami}
\subsubsection{Dieta z manj maščobami in brez omejitvije mase}
Zamaščena jetra? Ni problema! Naši eksperti (opice za računalnikom) so pripravili dieto
ki bo vaša jetra očistila v trenutku. Rezultati so prikazani na sliki (\ref{fig:fat-no-mass}). \\

\begin{figure}[H]
    \centering
    \makebox[\textwidth][c]{%
    \includegraphics[width=1.2\textwidth]{../Images/diet-model_min-fat-no-weight-visual.png}
    }
    \caption{Dieta z manj maščobami in brez omejitvije mase}
    \label{fig:fat-no-mass}
\end{figure}

Enkratno. Ena gajba grozdja na dan in boste kot novi. Šalo na stran je zanimivo, da je za naše
osnovne omejitve očitno grozdje dovolj hranilno v vseh kategorijah, če ga poješ dovolj. To je torej
prava space-age hrana. Or rather nek koncentrat grozdja. \\

\newpage
\subsubsection{Dieta z manj maščobami z dodatnimi omejitvami}
Ojoj zdravnik vam je rekel, da imate zamaščena jetra zaradi prekomernega pitja alkohola. Pa kolega, 
res ni problem. Naši pridni eksperti imajo tudi za to popolno dieto. Rezultati so prikazani na
sliki (\ref{fig:fat-add-no-mass}). \\

\begin{figure}[H]
    \centering
    \makebox[\textwidth][c]{%
    \includegraphics[width=1.2\textwidth]{../Images/diet-model_min-fat-add-no-weight-visual.png}
    }
    \caption{Dieta z manj maščobami z dodatnimi omejitvami}
    \label{fig:fat-add-no-mass}
\end{figure}

Saj vete kako pravijo eh? \textit{Klin se s klinom zbija, kaj ti škodi še liter vina?}. You heard 
correctly, efekte svoje crippling zasvojenosti z alkoholom lahko po naši briljantni dieti odpravite
s tem, da pijete še več alkohola. Pri tej dieti nikoli ne bote lačni saj je kalorično zelo učinkovita 
in garantiramo vam, da se vam bo zdel svet takoj lepši. Zabavno mi je, da se spet pojavi grozdje, le 
v drgačni obliki. Očitno je, da je grozdje res superhrana.\\

\subsubsection{Dieta z manj maščobami in omejitvijo mase}
Zaradi več pritožb strank, da se počutijo, kot da se prenajedajo (in morebitno kakšne smrti zaradi
zastrupitve z alkoholom) smo našim opicam.. ah mislim ekspertom razložili približno koliko banan lahko
tipičen človek poje na dan. Skovali so dieto, ki je prikazana na sliki (\ref{fig:fat}). \\



Ah ja, fantastično. Solata z marmelado je v nekaterih državah verjetno že tradicionalna jed in lepo je, 
da naši eksperti pustijo tudi nekaj lean mesa, kot je puran. Opazka tu je, da se grozdje sploh ne pojavi več.
Očitno je super le za neomejeno dieto, a v bolj realističnem primeru je gostota hranilnih snovi prenizka. \\

\subsubsection{Dieta z manj maščpbami, omejitvijo mase in dodatnimi omejitvami}
Za naše zveste stranke smo prejšno uspešnico diete z manj maščobami še malo izboljšali. Glavna pritožba
je bila, da vsebuje bistveno preveč sladkorja. Zdaj inzulin se menda lahko vedno samo vbrizga v žilo,
tako da to ni res problem, ampak vseeno. Rezultati so prikazani na sliki (\ref{fig:fat-add}). \\

\begin{figure}[H]
    \centering
    \makebox[\textwidth][c]{%
    \includegraphics[width=1.2\textwidth]{../Images/diet-model_min-fat-add-visual.png}
    }
    \caption{Dieta z manj maščobami, omejitvijo mase in dodatnimi omejitvami}
    \label{fig:fat-add}
\end{figure}

S to izboljšavo boste jedli še bolj zdravo saj boste lahko zaužili še $75$ g fižola več. Žal pa 
se bo treba odpovedati večini purana. Ampak ker so eksperti milostljivi so ga pustili za eno čajno 
žličko, da je za okus na koncu. Kakšnih drastičnih sprememb ni, le malo smo prerazporedili količine
hrane..

\begin{quote}
    \centering
    \textit{Tok. Tok. Dum. FBI open up!}
\end{quote}

\begin{figure}[H]
    \centering
    \makebox[\textwidth][c]{%
    \begin{minipage}{0.70\textwidth}
        \centering
        \includegraphics[width=\linewidth]{./fbi-1.png}
    \end{minipage}%
    \hspace{0.1\textwidth}% 
    \begin{minipage}{0.70\textwidth}
        \centering
        \includegraphics[width=\linewidth]{./fbi-2.png}
    \end{minipage}%
    }
\end{figure}

Oh shit aaaa hitro skrij opice. Beživa!

\subsection{Opmizacija cene}
\subsubsection{Čim cenejša dieta, ki zadosti osnovnim potrebam}
Poiščimo čim cenejšo dieto, ki zadostuje osnovnim potrebam kar se tiče hranilnih 
snovi. Rezultat je prikazan na sliki (\ref{fig:eur}).

\begin{figure}[H]
    \centering
    \makebox[\textwidth][c]{%
    \includegraphics[width=1.2\textwidth]{../Images/diet-model_min-eur-no-weight-visual.png}
    }
    \caption{Čim cenejša dieta}
    \label{fig:eur}
\end{figure}

Hja.. pravzaprav je kot nekako pričakovano. Ovseni kosmiči in mleko sta res med cenejšima 
izdelkoma in za zagotovitev soli še ena majhna porcija pomfrija, ki je tudi še kar cenen. Kar je
tudi vredno omembe je, da so kalorije za takšno dieto takoj višje, kar je znan efekt v dandanašnjem 
svetu. Dejansko je ceneje jesti nezdravo hrano.

\subsubsection{Čim cenejša dieta, z dodatnimi zahtevami}
Poglejmo še kaj bi dobili, če bi zraven še upoštevali še naše dodatne zahteve.
Rezultat je prikazan na sliki (\ref{fig:eur-add}).

\begin{figure}[H]
    \centering
    \makebox[\textwidth][c]{%
    \includegraphics[width=1.2\textwidth]{../Images/diet-model_min-eur-add-visual.png}
    }
    \caption{Čim cenejša dieta z dodatnimi zahtevami}
    \label{fig:eur-add}
\end{figure}

Kar smo zdaj dobili je hranilno bolj popolno, a bi bilo na dolgi rok, vsaj za večino ljudi,
škodljivo saj vsebuje preveč kalorij.

\subsubsection{Čim cenejša dieta, z dodatnimi zahtevami in omejitvijo kalorij}
Kot rečeno prejšnja dieta vsebuje preveč kalorij, torej lahko to poskusimo popraviti, tako da 
omejimo kalorije. Rezultat je prikazan na sliki (\ref{fig:eur-add-cal}).

\begin{figure}[H]
    \centering
    \makebox[\textwidth][c]{%
    \includegraphics[width=1.2\textwidth]{../Images/diet-model_min-eur-cal-visual.png}
    }
    \caption{Čim cenejša dieta z dodatnimi zahtevami in omejitvijo kalorij}
    \label{fig:eur-add-cal}
\end{figure}

In tu smo dobili nekaj kar je popolnoma uporabno. Vsaj meni se zdi upgrade no-brainer, ker 
z to dieto lahko jemo po omejitvi kalorij in to le za dober cent na dan več. Zdaj vprašanje je bolj 
če bi ta cent razlike se res ohranil, ker se običajno različne količine hrane v resnici
prodajajo po različnih cenah, torej če bi npr. kupili manj mleka, bi bilo to verjetno dražje. 

\subsubsection{Namenski bon za $15$ EUR}
Od šefa v službi smo prejeli namenski bon za $15$ EUR v naši najljubši živilski trgovini. Kaj 
bi bilo najbolj optimalno, da kupimo. Rezultati osnovnih zahtev so prikazani na sliki 
(\ref{fig:eur-budget}).

\begin{figure}[H]
    \centering
    \makebox[\textwidth][c]{%
    \includegraphics[width=1.2\textwidth]{../Images/diet-model_min-eur-budget-visual.png}
    }
    \caption{Namenski bon za $15$ EUR}
    \label{fig:eur-budget}
\end{figure}

Ah ja tu bi se zabavali. Tako dieto bi vam priporočal moj daljni stric iz devete vasi v 
Ameriki (za katerega sem slišal, da so ga ravno zaprli.. nekaj z nekimi opicami in reklamami).
Kakorkoli, tu se spet pojavi grozdje, ker nisem dal omejitve na maso. Overall bi sicer bila to 
kar slaba dieta, ampak bi bil človek mogoče preveč pijan, da bi opazil. \\

\subsubsection{Namenski bon za $15$ EUR z dodatnimi zahtevami in omejitvijo mase}
Kaj pa če bi zraven še upoštevali naše dodatne zahteve in omejili maso. Rezultati so prikazani
na sliki (\ref{fig:eur-budget-add}).

\begin{figure}[H]
    \centering
    \makebox[\textwidth][c]{%
    \includegraphics[width=1.2\textwidth]{../Images/diet-model_min-eur-add-budget-visual.png}
    }
    \caption{Namenski bon za $15$ EUR z dodatnimi zahtevami in omejitvijo mase}
    \label{fig:eur-budget-add}
\end{figure}

Ej takšno dieto bi pa tudi jaz z veseljem jedel. Losos je super, brokoli je mega in edamec 
je tudi fin. Zraven pa skuhaš kašo iz ovsenih kosmičev. To je verjetno najbolj kalorično 
učinkovita dieta, ki sem jo dobil thus far, saj je hrane vse skupaj manj kot kilogram, pa 
kljub temu presežemo $2000$ kcal. Grozno kako hitro se kalorije naberejo. \\


\section{Komentarji in izboljšave}

\newpage
\bibliographystyle{unsrt}
\bibliography{sources}
\end{document}
