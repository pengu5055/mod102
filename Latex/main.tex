\documentclass[a4paper]{article}
\usepackage[utf8]{inputenc}
\usepackage[slovene]{babel}
\usepackage{graphicx}
\usepackage{hyperref}
\usepackage[nottoc]{tocbibind}
\usepackage{caption}
\usepackage{subcaption}
\usepackage{amsmath}
\usepackage{ dsfont }
\usepackage{siunitx}
\usepackage{multimedia}
\usepackage[table,xcdraw]{xcolor}
\setlength\parindent{0pt}

\newcommand{\ddd}{\mathrm{d}}
\newcommand\myworries[1]{\textcolor{red}{#1}}
\newcommand{\Dd}[3][{}]{\frac{\ddd^{#1} #2}{\ddd #3^{#1}}}

\begin{document}
\begin{titlepage}
    \begin{center}
        \includegraphics[]{logo.png}
        \vspace*{3cm}
        
        \Huge
        \textbf{Linearno programiranje}
        
        \vspace{0.5cm}
        \large
        2. naloga pri Modelski analizi 1

        \vspace{4.5cm}
        
        \textbf{Avtor:} Marko Urbanč (28232019)\ \\
        \textbf{Predavatelj:} prof. dr. Simon Širca\ \\
        \textbf{Asistent:} doc. dr. Miha Mihovilovič\ \\
        
        \vspace{2.8cm}
        
        \large
        17.10.2023
    \end{center}
\end{titlepage}
\tableofcontents
\newpage
\section{Uvod}
Linearna optimizacija je zadnje čase zelo vroča tema, tako da me veseli, da sem se je lahko lotil tudi sam. 
Linearno programiranje oz. linearna optimizacija je metoda za iskanje maksimuma ali minimuma linearnega izraza,
ki je podvržen linearnim omejitvam. Omenjen linearni izraz je funkcija več spremenljivk, ki najpogosteje ovrednoti
oz. oceni neko količino, ki jo želimo optimizirati. To je lahko npr. dobiček, strošek, količina proizvodnje, itd.
zato to funkcijo imenujemo \textit{cost function} oz. \textit{objective function}. Linearno programiranje je torej
metoda, ki nam omogoča, da z linearno funkcijo in linearnimi omejitvami poiščemo optimalno rešitev. Uporablja v 
različnih panogah, kot so ekonomija, logistika, telekomunikacije, transport, kot pa tudi v znanosti, kot je npr. 
fizika. \\

Torej če si to pogledamo v matematični notaciji imamo našo cost funkcijo $f(x_1, x_2, \dots, x_n)$, 
definirano kot neko linearno kombinacijo spremenljivk:

\begin{equation}
    f(x_1, x_2, \dots, x_n) = c_1x_1 + c_2x_2 + \dots + c_nx_n\>.
\end{equation}

in set vezi, ki so pogosto izražene kot neenačbe:

\begin{equation}
    \begin{split}
        a_{11}x_1 + a_{12}x_2 + \dots + a_{1n}x_n &\leq b_1\>, \\
        a_{21}x_1 + a_{22}x_2 + \dots + a_{2n}x_n &\leq b_2\>, \\
        \vdots \qquad \qquad \quad\\
        a_{m1}x_1 + a_{m2}x_2 + \dots + a_{mn}x_n &\leq b_m\>. \\
    \end{split}
\end{equation}

To je pravzaprav to kar se tiče matematičnega opisa osnovnega problema. Seveda so potem izvedenke postopkov
oz. algoritmov malo bolj zapletene, ampak to žal ni namen te naloge. 




\section{Naloga}

\section{Opis reševanja}

\section{Rezultati}


\section{Komentarji in izboljšave}

\newpage
\bibliographystyle{unsrt}
\bibliography{sources}
\end{document}
